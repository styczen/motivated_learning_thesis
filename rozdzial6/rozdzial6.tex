\chapter{Agent w nieznanym środowisku}
\label{praktyka_ml}

Po zapoznaniu się z zagadnieniami związanymi z uczeniem motywowanym, środowiskiem
symulacyjnym, należało zdefiniować środowisko w jakim robot mobilny może się poruszać.
Jako robot testowy, użyto bardzo popularnej platformy Turtebot3 w wersji Waffle
(\cite{turtebot3}).

Ze względu na brak dostępu do prawdziwego robota oraz problemów jakie mogłoby sprawiać
zaprojektowanie środowiska w skali 1 : 1, zdecydowano o symulacyjnym rozwiązaniu.
Omawiane w poprzednich rozdziałach Gazebo, RViz i RQT zostały użyte do zaprojektowania
środowiska, sterowaniu robota i wizualizacji aktualnej pozycji, stanu wewnętrznego 
robota oraz stanu środowiska.

Uczenie motywowane jest jeszcze mało zbadanym zagadnieniem, a celem tej pracy jest
zbadanie wykorzystania algorytmów uczenia motywowanego na robocie mobilnym.
Definiowane są dla niego pewne prymitywne bóle i akcje jakie może on wykonać w danej
chwili czasu. W pierwszym rozwiązaniu ograniczono do zdefiniowania kilku prymitywnych bóli.
Agentem jest robot mobilny, który ma możliwość wykonania ruchu na mapie